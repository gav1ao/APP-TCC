\chapter{Introdução}

%\section{Motivação}

%% FIXME: Adicionar referências
Na década de 1990, o Brasil passou por um período de reajustes de preços praticamente diários causados por inflação alta decorrente de uma péssima gestão econômica no País. Caso um indivíduo efetuasse uma compra durante o período diurno, no vespertino, esse mesmo produto poderia estar com seu valor duplicado ou até mesmo triplicado. Muitos anos após esse período conturbado na economia, o brasileiro ainda possui o desejo de efetuar compras através do menor preço, o que pode ser comprovado com o aumento do número de compras durante a \textit{Black Friday}.

Ao efetuarmos uma compra no comércio local, a nota fiscal referente à compra dos produtos é emitida em conformidade com a lei vigente. Atualmente, a nota fiscal possui uma chave de acesso a qual permite o indivíduo acessar o mesmo conteúdo impresso em uma plataforma disponibilizada pela Secretaria de Fazenda de cada estado da federação. Essa chave é fornecida de duas formas, através de um \textit{QRCode} ou de um código composto por 44 algarismos numéricos. A partir dessas informações, é possível acessar o conteúdo da compra em um endereço \textit{online}.

%\section{Objetivos}

O trabalho desenvolvido, aqui apresentado, tem como objetivo principal ser uma ferramenta de auxílio a população para que possam efetuar compras, dentro do possível, com o menor preço no mercado local. Para tal, foi desenvolvido um aplicativo em que uma pessoa poderá instalar em seu aparelho móvel para que possam registrar e recuperar as informações de compras efetuadas em mercados. Para isso ser possível, é feito com o auxílio da nota fiscal em sua versão digital, isto é, a Nota Fiscal Eletrônica (NF-e). O aplicativo desenvolvido faz o usufruto dessa informação para poder recuperar as informações contidas nas notas fiscais e armazenar em um banco de dados. Com isso, é possível efetuar a recuperação posterior das informações de um determinado produto.

%\section{Público Alvo}

O público alvo consiste em pessoas que possuem um aparelho celular do tipo \textit{Smartphone} e com acesso à internet, pois todo processamento da informação é feito em um meio externo ao telefone móvel, assim as informações são processadas em um servidor no qual o aplicativo se comunica através da internet.

No Capítulo 2, são apresentadas as tecnologias disponíveis para o desenvolvimento da aplicação como também a definição de alguns conceitos necessários para a melhor compreensão do trabalho. O desenvolvimento do aplicativo e do servidor, necessário para o funcionamento do primeiro, é abordado no Capítulo 3. Já o Capítulo~4, são feitas as apresentações das telas do aplicativo desenvolvido como também algumas observações efetuadas durante o período do desenvolvimento. Por fim, no Capítulo 5 temos a conclusão do projeto desenvolvido.
