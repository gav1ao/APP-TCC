\chapter{Introdução}

\section{Motivação}

%% FIXME: Adicionar referências
Na década de 1990, o Brasil passou por um período de ajustes de preços praticamente diários durante a época de alta da inflação. Caso um individuo efetuasse uma compra durante o período diurno, pela tarde, esse mesmo produto poderia estar com seu valor duplicado ou até mesmo triplicado.\\

%% FIXME: Melhorar texto. Efetuar conexão com texto anterior. Adicionar dados estatísticos.
Muitos anos após esse período conturbado na economia, o brasileiro ainda possui o desejo de efetuar compras através do menor preço, o que pode ser comprovado com o aumento do número de compras durante a \textit{Black Friday}.\\

\section{Objetivos}

O trabalho desenvolvido, aqui apresentado, tem como objetivo principal ser uma ferramenta de auxílio diário a pessoas que possuem o desejo de efetuar compras sempre com o menor preço possível.

\section{Funcionamento Básico}

O trabalho trate-se de um aplicativo que uma pessoa poderá instalar em seu celular e usufruir os recursos que o mesmo possibilita.\\
O aplicativo permite ao usuário registrar e recuperar as informações previamente salvas. O registro é feito com o auxílio da nota fiscal em sua versão digital, isto é, a Nota Fiscal Eletrônica (NF-e).\\

% FIXME: http://www.nfe.fazenda.gov.br/portal/perguntasFrequentes.aspx?tipoConteudo=E4+tmY+ODf4=
% Melhorar definição e adicionar citação
Conforme é definido no site da Receita Federal, a NF-e é...\\

Toda nota fiscal emitida em conformidade com a lei, possui uma chave de acesso permite ao indivíduo acessar o mesmo conteúdo impresso em uma plataforma disponibilizada pela Secretaria de Fazenda de cada estado da federação. Essa chave é fornecida de duas formas, através de um \textit{QRCode} ou código composto por 44 algarismos numéricos. De posse dessa informação, a pessoa conseguirá acessar o conteúdo da compra em uma página da web.\\

O aplicativo desenvolvido faz o usufruto dessa informação para poder recuperar as informações contidas na nota fiscal e armazenar em um banco de dados para possibilitar a recuperação das informações de um determinado produto.

\section{Público Alvo}

O público alvo do trabalho desenvolvido consiste em pessoas que possuem um aparelho celular moderno do tipo \textit{smartphone} e com acesso à internet, pois todo processamento da informação é feito em um meio externo ao telefone móvel. As informações são processadas em um servidor que o aplicativo se comunica através da internet.

\section{Visita Guiada}
%%FIXME: Escrever essa secção
\begin{itemize}
    \item Falar sobre o que será abordado nos outros capítulos
    \item Corrigir o nome da secção
\end{itemize}



\subsection{Seção terciária}

\subsubsection{Seção quartenária}

\subsubsubsection{Seção quinária}