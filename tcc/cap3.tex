\chapter{Desenvolvimento}

Este capítulo tem o objetivo de descrever o aplicativo desenvolvido neste trabalho para dar suporte à compra de produtos em mercados locais. Será inicialmente descrito o acesso às informações referentes a compra de produtos, assim como os aspectos envolvendo o tratamento e o armazenamento desses dados, como também o mecanismo de busca e visualização das consultas aos produtos. Além disso, serão apresentados alguns recursos e serviços utilizados durante o período de desenvolvimento.

\section{Obtenção das informações}
\label{obtencaoInformacoes}

Conforme abordado no capítulo anterior ao longo da \autoref{secNfce}, cada estado fornece um meio online para acesso da versão eletrônica das notas fiscais, com isso, não existe um sistema único para consulta de diversas notas de estados diferentes. Devido a essa peculiaridade, a versão desenvolvida descrita nesse projeto, possui suporte a somente notas emitidas no estado Fluminense. Todavia, a adição do suporte a outros estados é facilitada, pois a inclusão já foi prevista durante o desenvolvimento.

A partir do sistema disponibilizado pelas secretarias de fazenda, o usuário poderá acessar a versão online da nota fiscal através de duas formas: por meio de um site específico junto de uma chave de acesso ou pelo endereço web contido no QRCode.

Independente da forma que foi efetuado o acesso, após o término da solicitação, o usuário é redirecionado para uma página que contém a versão online da nota fiscal. O conteúdo dessa página é analisado e armazenado. A técnica utilizada para a análise desse conteúdo é o Web Scraping.

Portanto, após o usuário solicitar a adição de uma nota fiscal, no aplicativo, será feita uma simulação de uma dessas formas de acesso, o que resultará em uma requisição ao próprio site da Secretaria de Fazenda, consequentemente, será feito o Scraping da página assim que o resultado estiver disponível.

Ainda, vale destacar que o conteúdo da nota fiscal pode ser obtido de outra forma que se dá através do uso de técnicas de processamento de imagem. Para isso, a NFC-e deverá ser escaneada e por meio da imagem resultante poderá ser feito o reconhecimento dos caracteres e assim obter a informação necessária para análise. A técnica de reconhecimento de texto OCR, que foi brevemente apresentada na \autoref{secao-captcha-definicao}, também poderá ser utilizada nesse caso, no entanto, será necessário que efetue a aquisição de uma imagem em alta qualidade, o que restringe uso nos modelos de entrada de celulares por possuírem câmeras com qualidade inferior.

\subsection{Simulação da solicitação}

Conforme dito anteriormente, independente da forma de acesso à nota, o mesmo conteúdo poderá ser obtido, porém os processos de reprodução dessas formas possuem algumas diferenças entre si.

A solicitação através da chave de acesso é feita de forma a simular o acesso do usuário ao site original, isto é feito com o auxílio da biblioteca Puppeteer para o Node.js. Por meio dessa biblioteca, é possível simular a navegação em um navegador assim como seria feito por um indivíduo. Desta forma, o site disponibilizado para consulta das notas é acessado com o auxílio dessa ferramenta, no entanto, para obter o conteúdo da nota é necessário passar por um desafio cognitivo (que é o CAPTCHA) e com isso o texto referente a esse desafio também deverá ser inserido para que se possa comprovar que a solicitação está sendo feito por um humano e não por uma ferramenta automatizada, logo, esse desafio é repassado para o usuário do aplicativo para que o mesmo solucione esse teste e assim a requisição poderá ser validada.

Caso a nota fiscal já esteja disponível no sistema da SEFAZ e a informação necessária ao desafio esteja correta, o conteúdo será disponibilizado, caso contrário, o usuário será notificado que houve um erro durante o cadastro da NFC-e na própria aplicação. Caso seja identificado o motivo desse erro, será mostrada uma breve mensagem da causa do problema, mas caso não seja possível efetuar a identificação do problema, uma mensagem genérica será mostrada.

Já para reprodução com o QRCode, é possível obter o conteúdo diretamente sem a necessidade da solução de um desafio cognitivo, todavia, ainda é necessária a simulação da navegação com o Puppeteer. Assim como na forma anterior, caso a NFC-e não seja encontrada no sistema da SEFAZ, o usuário também será notificado que a mesma não se encontra disponível. Vale destacar que para a adição da NFC-e com o QRCode, é necessário que o usuário possua um smartphone com câmera. Caso qualquer erro não previsto durante o processamento da requisição ocorra ou o sistema da SEFAZ esteja inoperante, o usuário também será notificado sobre a impossibilidade da adição com uma mensagem de erro genérica. Tal situação poderá ocorrer independente da forma em que a solicitação foi efetuada.

\subsection{Análise dos dados}

A análise dos dados é feita com o auxílio da ferramenta Cheerio\cite{cheerio}, que é um módulo baseado no jQuery, um Framework Web bastante popular do JavaScript\cite{stackOverflowRanking}, para utilização com o Node.js. Com o conteúdo da página, é possível obter cada informação individualmente, logo é possível saber quais produtos foram comprados como também o preço unitário e a quantidade comprada de cada produto. Além disso, também é possível armazenar os dados do local da compra como o nome do estabelecimento e o seu endereço. Por fim, dados gerais da compra também podem ser recuperados como a data e a hora em que foi realizada.

A recuperação da informação contida na página é possível pois o conteúdo é construído com a linguagem HTML e através dos elementos que compõe a página, é possível identificá-los e efetuar a separação da informação.

\section{Armazenamento das informações}
\label{armazenamentoInfo}

% TODO: Modelagem dos dados
% https://medium.com/@gpanassol/como-posso-fazer-modelagem-de-dados-em-mongodb-ea61268ee10b
% https://felipetoscano.com.br/modelando-dados-no-mongodb/
% file:///home/vitor/%C3%81rea%20de%20Trabalho/[NoSQL%20e%20MongoDB%20Users]%20Modelagem%20de%20Dados%20para%20BD%20Orientado%20a%20Documentos.pdf
% https://www.univates.br/bdu/bitstream/10737/1674/1/2017MarcelKober.PDF
% https://www.monografias.ufop.br/bitstream/35400000/1963/6/MONOGRAFIA_An%C3%A1liseProjetosBanco.pdf
Com os dados provenientes da NFC-e já processados, ou seja, analisados e separados, esse conteúdo resultante é armazenado em um banco de dados para que a informação seja disponibilizada posteriormente para consultas a partir da aplicação.

Esse conteúdo é armazenado em um banco de dados e o tipo utilizado pela aplicação é o MongoDB, que é um banco de dados não-relacional onde os dados são armazenados em formato de documento. Como a aplicação está sendo feita em JavaScript, foi optado em utilizar esse banco devido à conveniência resultante do uso junto ao Json.\cite{mongoDBDefinition}

Além disso, esse tipo de banco é altamente escalável, isto é, caso seja necessário mais espaço para o armazenamento dos dados, o conteúdo pode ser distribuído em diversas máquinas sem afetar o desempenho geral da aplicação.

\subsection{O armazenamento remoto e os dados compartilhados}

Os dados provenientes do processamento das notas fiscais poderiam ser armazenados nos celulares dos usuários do aplicativo, o que resultaria no consumo de espaço de armazenamento que também é compartilhado com outros aplicativos e o sistema operacional do aparelho móvel. Embora seja uma alternativa viável para o cliente, e possivelmente rápida, uma vez que a recuperação não sofrerá latência de rede, a longo prazo talvez não seja considerada muito atrativa. A justificativa dessa afirmação é o fato de que para cada nota adicionada uma parcela do armazenamento é consumida, ademais, não existe uma rotina de limpeza dos dados mais antigos.

Apesar da implementação dessa rotina ser simples, restringiria uma possível melhoria ao produto, por exemplo, a disponibilização da série histórica de preços dos produtos a longo prazo.

Outro ponto que pode ser destacado, é que com os dados concentrados no dispositivo de um indivíduo, os mesmos estão propensos a exclusão, que pode ser causada por diversas situações, por exemplo, a eliminação decorrente da remoção do aplicativo, a formatação do aparelho devido à restauração ao modo de fábrica ou até mesmo uma pane no aparelho em certos casos.

Após a observação dos pontos acima, a utilização de uma solução que envolva o armazenamento da informação em um meio externo torna-se viável, pois a segurança da informação é melhor garantida, uma vez que a cópia de segurança é garantida do lado do servidor desonerando o cliente dessa responsabilidade.

Uma vantagem garantida ao preservar os dados dessa forma, é que uma base maior pode ser criada pois as notas de todos os usuários poderão estar guardadas em único local, o que acarreta em um compartilhamento de informação e a criação de uma base de dados colaborativa, aonde um usuário A adiciona sua NFC-e e os dados dos produtos estarão disponíveis para um usuário B, e caso esse último quisesse saber o custo de um produto comprado pelo usuário A, poderia consultá-lo por meio do aplicativo sem a necessidade de ir ao mercado utilizado pelo usuário A.

Vale destacar que, apesar da utilização de uma base compartilhada, não há como um usuário ter acesso ao valor gasto em uma compra de outro usuário, uma vez que as notas são adicionadas sem a necessidade de um cadastro e somente é possível obter a informação de produtos, e não de uma compra como um todo.

Abaixo, temos um exemplo de uma nota fiscal real no formato que se encontra armazenada no banco de dados. Vale destacar que está sendo salvo no formato Json que é um tipo de documento.

\usemintedstyle{manni}
\inputminted{octave}{exemplo-nfce.json}

\newpage
% TODO: Melhorar seção
\section{Disponibilização dos dados}

% Dependnecia de internet vs offilien
% Contraatacar com o processamento

% Processamento no proprio aparelho
% Usar do processador / bateria/internet de qualidade /

Os dados são disponibilizados com o auxílio do servidor principal da aplicação no qual possui uma conexão com o servidor em que se encontra o banco de dados. Através do Framework Express é possível criar de uma forma flexível uma API, que permite a comunicação com outras aplicações. Com isso, é disponibilizada uma URL que permite a consulta dos produtos salvos na base de dados.

A consulta é feita através do nome do produto, e esse nome pode estar acentuado corretamente, erroneamente ou até mesmo sem acentos, que os mesmos dados serão retornados. Os dados retornados contêm o nome do produto da forma que consta na nota fiscal, o nome do estabelecimento no qual a compra foi efetuada, a data e a hora da compra. Para cada produto compatível com a consulta, é retornado o último registro salvo para cada estabelecimento para que o usuário sempre possa ter o preço mais recente.

Vale ressaltar que é possível refinar a cidade em que a consulta deverá ser feita, assim como retornar os registros de todas as cidades disponíveis.

\subsection{Ausência de padronização nos nomes}

Um problema que pode ser observado a partir de uma simples consulta com o aplicativo é a ausência de padronização nos nomes dos produtos, haja vista que cada mercado cadastra em seu banco de dados as informações referentes aos produtos a seu bel-prazer. Como também, devido ao espaço limitado nas notas fiscais impressas, faz com que ocorra uma abreviação nos produtos.

Uma alternativa aos nomes que poderia ser utilizada, seria a utilização do código de barras presente em cada produto, por outro lado, para aqueles produtos que são vendidos por peso, não há uma uniformidade no código desses produtos, ficando a critério de cada estabelecimento definir um código.

Visto esses problemas descritos, uma solução híbrida pode ser utilizada, isto é, dado um nome de um produto, com auxílio de serviços de terceiros, é possível recuperar o código de barras para o produto pesquisado, com isso, poderia efetuar a busca tanto pelo nome quanto pelo código de barras. Essa solução é sugerida como uma melhoria futura para a aplicação.

\section{Hospedagem}

Conforme descrito nas seções \ref{obtencaoInformacoes} e \ref{armazenamentoInfo}, um servidor torna-se necessário tanto para o processamento e disponibilização dos dados quanto para o armazenamento dos mesmos. Com isso um serviço de hospedagem faz-se necessário para ambos os casos.

\subsection{Backend}

Como todo processamento dos dados é feito em meio externo ao telefone móvel do usuário, logo, a utilização de um servidor é necessária e com isso será necessário a utilização de máquinas para poder executar o código desenvolvido para esse propósito. Um computador pessoal de baixo custo atende esse requisito, no entanto, à medida que o uso do aplicativo tornar-se mais frequente e a base de clientes for aumentando, uma máquina com mais recursos será imprescindível.

O uso de uma máquina física para esses propósitos é uma solução viável, no entanto, existem serviços que fornecem recursos de hospedagem a um custo acessível e caso elegível, e em alguns casos até mesmo de forma gratuita.

Uma das principais vantagens do uso de serviços de hospedagem é que o serviço não será interrompido caso haja queda de energia ou até mesmo seja perdida a conexão com a internet. Um dos serviços mais populares que oferecem tais recursos de hospedagem é o Heroku\cite{heroku}, que fornece desde uma máquina simples gratuita até serviços mais robustos focados para o uso de alto fluxo de dados como ocorre em grandes corporações.

Durante os primeiros meses de desenvolvimento, foi feito o uso do plano gratuito para testes, no entanto, caso não houvesse nenhuma iteração com a API desenvolvida durante um período de 30 minutos, a mesma ficaria inativa. Diante desse fato, a solicitação de adição de notas tornava-se mais lenta pois além do tempo de processamento dos dados, o tempo de inicialização aumentava a latência em que a resposta era retornada ao usuário. Todavia, a iteração ainda poderia ser feita normalmente.

% TODO: Add ref https://education.github.com/pack
A empresa GitHub, que permite que desenvolvedores e empresas armazenem e divulguem seus códigos, possui um programa para estudantes que permite que os mesmos acessem e usufruem de recursos pagos sem gerar nenhum ônus. Esse programa é conhecido como GitHub Student Developer Pack\cite{githubStudentPack} que é feito em parceria com outras empresas do ramo da tecnologia.

O Heroku é um serviço que está incluso nessa parceria fornecendo uma máquina paga com muitos recursos sem nenhum custo ao estudante participante desse programa por um período limitado. Como o autor é elegível, o uso dessa máquina paga está sendo feito para os testes simulando o uso real de um servidor que pode ser utilizado em produção, isto é, recebendo altas cargas de requisições.

Outras empresas que poderiam ser utilizadas seriam a DigitalOcean, Locaweb, Hostgator e quaisquer outras com recursos semelhantes.

\subsection{Banco de dados}

Assim como o processamento dos dados, o armazenamento dos dados também é feito em meio externo ao celular do usuário, para isso é feito o uso de serviços em nuvem. Durante o desenvolvimento foi utilizado o serviço MongoDB Atlas Cloud através do plano gratuito. Com isso é fornecida uma máquina com capacidade de armazenamento de 512mb e com a memória RAM compartilhada entre outros usuários, isto é, vários bancos são executados em uma mesma máquina.

Caso seja necessário o uso de uma máquina dedicada ou com maior capacidade, essa mesma empresa fornece outros planos pagos para aplicações que necessitem de maior desempenho ou armazenamento.

Como alternativa ao serviço utilizado, algumas empresas que possuem planos com máquinas dedicadas também poderão ser utilizadas para a execução do serviço do banco.

\section{Desenvolvimento do aplicativo}\label{desenvApp}

O desenvolvimento do aplicativo foi feito utilizando uma tecnologia de desenvolvimento multiplataforma com código nativo, para isso foi utilizado o React Native. Com isso, é possível desenvolver um único aplicativo e publicar para as duas plataformas mais populares, isto é, o iOS e o Android\cite{reactNativeAbout}. Sendo feito dessa forma, houve uma otimização no tempo de desenvolvimento.

A justificativa para a escolha do uso do React Native, primeiramente é pelo fato de permitir um desenvolvimento multiplataforma com código nativo, em acréscimo, estava melhor colocado no ranking elaborado pelo StackOverflow, presente na \autoref{tab-stack-overflow-framework-tools}, quando comparado com seus concorrentes diretos. Além disso, caso futuramente sejam disponibilizado as mesmas funcionalidades presentes no aplicativo para um website, isso poderá ser feito sem muitas complicações, sendo necessárias poucas alterações.

\subsection{Expo}

O Expo é um Framework e uma plataforma para aplicações de React. Com isso, acaba por ser um conjunto de ferramentas e serviços, criados em torno do React Native e das plataformas nativas, que permite ao programador desenvolver e distribuir sua aplicação entre os sistemas móveis facilmente a partir do mesmo código desenvolvido em JavaScript.\cite{expo}

Além disso, o Expo proporciona uma abstração dos recursos de aparelhos móveis sem a necessidade de implementação de código nativo. Os recursos são tais como os contatos telefônicos, a câmera, acesso ao GPS do celular e outros. Com isso, é possível recuperar a informação proveniente desses recursos através do próprio código em JavaScript. Sendo assim, foi optado em utilizar essa ferramenta de forma a acelerar o desenvolvimento do aplicativo.\cite{expoOverview}

Como também é necessário o acesso à câmera para a leitura de QRCodes, tal ferramenta tornou-se indispensável, pois, por ser uma abstração de código nativo, é possível disponibilizar essa funcionalidade para ambas as plataformas sem complicações.

% Explicar a facilidade que o Expo proporciona
% Port para as aplicações

\subsection{Estilização}

De forma a acelerar o tempo de desenvolvimento do aplicativo, algumas ferramentas foram utilizadas para a estilização do aplicativo. Tais ferramentas são um conjunto de componentes prontos, isto é, já estilizados, bastando somente efetuar a adaptação para o projeto como o posicionamento e o tamanho.

Desses recursos, foram utilizados dois pacotes de estilos, o NativeBase e o React Native Paper. Ambos, fornecem estilos semelhantes aos nativos disponíveis em cada plataforma.

% NativeBase
% Native Paper

% Falar aqui dos templates usados com o React Native
% Vantagem de manter o estilo nativo para o Android e o IOS