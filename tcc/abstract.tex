%% Elemento obrigatório (Figura 15).
%% Consiste em uma tradução do resumo em português para uma
%% língua estrangeira (em inglês, ABSTRACT; em espanhol, RESUMEN;
%% em francês, RÉSUMÉ), em um único parágrafo, seguido das palavras-
%% -chave representativas do conteúdo do trabalho, na língua estrangeira
%% escolhida.
%% O resumo em outra língua também é precedido pela referência
%% do trabalho, substituindo-se o título em português pelo título na língua
%% estrangeira adotada.
%% No caso de teses, é possível incluir dois resumos em língua es-
%% trangeira.
%% A apresentação gráfica e a ordem dos elementos seguem a mes-
%% ma orientação do resumo em português.

\begin{resumo}[Abstract]
\begin{otherlanguage*}{english}

\noindent
\entradaAutor{}. \textit{\englishTitle{}}. 2020. \pageref{LastPage} f. Trabalho de Conclusão de Curso (Graduação em Engenharia de Computação) - Instituto Politécnico, Universidade do Estado do Rio de Janeiro, Nova Friburgo, 2020.
\vspace{\onelineskip}

\setlength{\parindent}{1.3cm}
The course completion work presented here aims to develop an application that can facilitate people's daily lives while shopping in supermarkets. The method used to obtain the information related to purchases is the data scrapping, and it parses the information available in the electronic version of the invoices issued in each purchase. Using a technique known as Web scraping, it is possible to carry out this data collection, thereby some information such as prices, places of each purchase as well as the dates on which they were carried out is stored, therefore, this recorded information can be accessed from the mobile application. The entire process of registering invoices and recovering the product data is done by the application itself and with some code executed on an external machine. In addition, the data collection and treatment process is presented, as well as the application interface. Finally, it should be noted that every code developed is functional, and it is a collaborative platform where each contribution of an individual will benefit himself as well as all other users.

\vspace{\onelineskip}
\noindent Keywords: Purchasing Facilitator. Invoice. NFC-e. Mobile application. Web scraping.

\end{otherlanguage*}
\end{resumo}