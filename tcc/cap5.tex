% Workaround para criação de capítulo não-numerado alinhado à margem
\chapter*{}
\noindent
\phantomsection{\MakeUppercase{\textbf{Conclusão}}}
\addcontentsline{toc}{chapter}{CONCLUSÃO}
\newline
\newline

Por fim, podemos destacar que o objetivo principal que era a construção de um aplicativo simples e colaborativo visando facilitar a compra de produtos no comércio local através do registro de informações referentes a essas compras e com a possibilidade da recuperação das informações referentes as mesmas foi alcançado.

Alguns pontos que devem ser observados é que se o aplicativo tornar-se popular, isto é, uma grande parcela da população tornar habitual o uso do mesmo durante suas compras, poderá acarretar em uma possível redução no lucro daqueles estabelecimentos que possuem produtos com um valor maior do que a média, no entanto, isso resultará em um benefício para os consumidores pois esses mesmos supermercados deverão efetuar a diminuição dos preços de seus produtos, todavia, o lucro dessas empresas poderá ser obtido através do grande volume de vendas.

O autor gostaria de destacar que possui a intenção de efetuar a manutenção da aplicação através de novas funcionalidades e suporte a outros estados da federação, além disso, o código desenvolvido é disponibilizado em um repositório público fazendo com que essa versão seja um projeto de código aberto.\\

\noindent
\phantomsection{\textbf{Trabalhos futuros}}
\addcontentsline{toc}{chapter}{Trabalhos futuros}
\\

Embora o aplicativo tenha atingido o objetivo principal, que era o cadastro das notas fiscais e a recuperação dos preços dos produtos contidos nas mesmas, novas funcionalidades podem ser incluídas de forma a melhorar ainda mais a experiência do usuário e facilitar mais ainda o cotidiano dos cidadãos. Como por exemplo, pode ser adicionado o suporte ao sistema de outros estados fazendo com que mais brasileiros possam utilizar o aplicativo não ficando limitado somente à população do Estado do Rio de Janeiro.

Além disso, melhorias poderão ser realizadas durante o processo de leitura do QRCode, isto é, uma vez que possa ser impresso em uma qualidade ruim poderá ocasionar problemas durante a leitura, fazendo com que a mesma não seja realizada com sucesso. De forma a contornar esse problema, poderá ser efetuado um pós-processamento da imagem após um tempo gasto estipulado anteriormente, com isso, ainda será possível efetuar a adição da nota sem a necessidade de digitação de textos.

Ainda durante o processo de leitura do QRCode, se após o processamento da imagem não for possível concluir a leitura e a obtenção da informação, poderá ser oferecido ao usuário a opção de adição através do código de acesso automaticamente caso o aplicativo detecte que há uma demora para efetuar a leitura.

Por fim, uma outra melhoria que poderá ser implementada, é uma outra forma de notificação ao usuário que sua nota foi processada fazendo com que o mesmo possa utilizar outros aplicativos não sendo mais necessário permanecer com uma das telas de cadastro abertas aguardando a resposta. Com isso, poderá ser adicionada mais de uma nota por vez, o que melhorá ainda mais a serventia do projeto.