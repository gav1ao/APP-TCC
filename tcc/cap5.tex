\chapter{Conclusão}

Por fim, podemos destacar que o objetivo principal que era a construção de uma aplicação para registro de informações referentes a compras e a possibilidade da recuperação da mesma posteriormente foi alcançado.

Alguns pontos que devem ser observados é que se o aplicativo tornar-se popular, isto é, uma grande parcela da população tornar habitual o uso do mesmo durante suas compras, poderá acarretar em uma possível redução no lucro daqueles estabelecimento que possuem produtos com um valor maior do que a média, no entanto, isso resultará em um benefício para os consumidores pois esses mesmo supermercados deverão efetuar a diminuição dos preços de seus produtos, todavia, o lucro dessas empresas poderá ser obtido através do grande volume de vendas.

O autor gostaria de destacar que possui a intenção de efetuar a manutenção da aplicação através de novas funcionalidade e suporte a outros estados da federação, além disso, o código desenvolvido será disponibilizado em um repositório público fazendo com essa versão seja um projeto de código aberto.